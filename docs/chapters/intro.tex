This document is going to describe a basic RISC-V datapath written in VHDL, starting from a simple non-pipelined architecture just to implement more complex features and instructions with a bottom-up approach.
When talking about RISC-V it is necessary to distinguish the two faces of the ISA:
\begin{itemize}
    \item \emph{Unprivileged ISA}: that defines the user-level instruction types, which mainly compose a program to be executed.
    \item \emph{Privileged ISA}: which defines exception management and operation modes (supervisor, hypervisor and user mode), interrupt handling and multiple other functionalities for operating system support.
\end{itemize}
The project will focus on partially implementing the base user-level instructions, defined in the unpriviledged ISA, defining at each stage of the execution what information in an instruction are used and how they affect the state of the sequential network, helping the explanation with behavioral simulations, and building the project in a way that can be later expanded with other features.
Thus, the first stage of this project will focus on implementing a datapath that executes some commands included in the User Instruction Set, in particular, for the first part of the project, the 32-bit Base Integer Instruction Set (RV32I).
The report will be divided into chapters for each of the following 5 stages of the CPU:
\begin{itemize}
\item \textbf{Instruction Fetch}
\item \textbf{Instruction Decode} 
\item \textbf{Instruction Execute} 
\item \textbf{Data Memory}
\item \textbf{Writeback} 
\end{itemize}
The completion of the first rudimental architecture, able to execute one instruction per clock cycle, has to be then optimized by introducting registers between each block, and thus achieving a \emph{pipelined architecture}, adding more complexity overall yet incrementing the performance of the circuit.\\
This addition although does not come without problems, a pipelined architecture introduces some inconsistencies that need to be addressed in order to have the architecture to remain fully-functional. In particular:
\begin{itemize}
    \item \textbf{Data Hazards}: Subsequent instructions that access the same registers, for both reading and writing, in different stages of the pipeline, need to have the most recent data available without having to wait for each previous instruction to complete its cycle of execution.
    \item \textbf{Control Hazards}: Pipelined architectures need to carefully manage jumps and, if needed, stop every other partial execution of any instruction that comes next, since there is no certainty that the flow of the program will continue in the same way as before the jump.
\end{itemize}
With everything fixed, the five stages can finally operate concurrently and thus actually benefitting from having a pipeline.\\
At the very end, having a synthetizable code does not mean that the code will follow the behavioral simulation, so the code will be implemented on a Nexys 4 DDR with an additional expansion for the core to interact with the 7-segments display and switches to actually see the content of the Data Memory and Register File, just to confirm that everything i working properly.\\
With this being said, the project will follow said procedure, while making available the code in Appendix B for the unpipelined datapath, Appendix C for the pipelined one, and Appendix D for the extra blocks for testing purposes.
