With a fully functional, syntetizable CPU, it has been shown how a simple RISC-V core can be designed in VHDL. However, this project, as of now, does not include a decent interaction between L1 cache (both I-cache and D-cache) and an external L2 cache or RAM; Also, only the RV32I is implemented (partially) and for any other expansion that can stem from the full ISA there is the need to extend the code in both the decoder and the ALU, although it should be much simpler to do so from this point on. \\
Adding features with a working base to start from should be easy, and the project could take another step further in the future; One good feature would be adding the capability for the processor to execute multiplications and divisions between integers, easy enough since VHDL introduces the synthetizable operators '*' and '/'.\\
One thing that won't be easy, though, is implementing floating-point operations, since it would require the introduction to a data type that follow the IEEE 754 standard for floating point numbers. The only way to avoid the problem would be having them encoded in VHDL using the data type \textbf{real}, which is not synthetizable, although it would make a good option for HDL designs that won't go further from simulation.\\