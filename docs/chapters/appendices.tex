\section{Appendix A: Instruction Tables}

\begin{table}[ht]
    \begin{center}
        \begin{tabular}{|c|c|c|c|c|}
            \hline
            class & funct7 & funct3 & instruction & operation \\
            \hline
            \multirow{10}{*}{OP} & 0000000 & 000 & ADD & addition\\
            & 0100000 & 000 & SUB & subtraction\\
            & 0000000 & 001 & SLL & left logic shift\\
            & 0000000 & 010 & SLT & set if less than\\
            & 0000000 & 011 & SLTU & set if less than unsigned\\
            & 0000000 & 100 & XOR & bit-wise EXOR\\
            & 0000000 & 101 & SRL & right logic shift\\
            & 0100000 & 101 & SRA & right arithmetic shift\\
            & 0000000 & 110 & OR & bit-wise OR\\
            & 0000000 & 111 & AND & bit-wise AND\\
            \hline
            \multirow{9}{*}{OP-IMM} & - & 000 & ADDI & addition\\
            & 0000000 & 001 & SLLI & left logic shift\\
            & - & 010 & SLTI & set if less than\\
            & - & 011 & SLTIU & set if less than unsigned\\
            & - & 100 & XORI & bit-wise EXOR\\
            & 0000000 & 101 & SRLI & right logic shift\\
            & 0100000 & 101 & SRAI & right arithmetic shift\\
            & - & 110 & ORI & bit-wise OR\\
            & - & 111 & ANDI & bit-wise AND\\
            \hline
            \multirow{3}{*}{STORE} & - & 000 & SB & addition\\
            & - & 001 & SH & addition\\
            & - & 010 & SW & addition\\
            \hline
            \multirow{5}{*}{LOAD} & - & 000 & LB & addition\\
            & - & 001 & LH & addition\\
            & - & 010 & LW & addition\\
            & - & 100 & LBU & addition\\
            & - & 101 & LHU & addition\\
            \hline
            JALR & - & 000 & JALR & addition\\
            \hline
            JAL & - & - & JAL & addition\\
            \hline
            LUI & - & - & LUI (pseudo instruction LI) & addition\\
            \hline
            AUIPC & - & - & AUIPC & addition\\
            \hline
            \multirow{6}{*}{BRANCH} & - & 000 & BEQ & branch if equal\\
            & - & 001 & BNE & branch if not equal\\
            & - & 100 & BLT & branch if lower than\\
            & - & 101 & BGE & branch if greater or equal\\
            & - & 110 & BLTU & branch if lower than unsigned\\
            & - & 111 & BGEU & branch if greater or equal unsigned\\
            \hline
        \end{tabular}
        \caption{RV32I core instructions}
        \label{table:core_instr}
    \end{center}
\end{table}

\section{Appendix B: Code segments of unpipelined datapath}

\subsection{Instruction Fetch [RV32I]}

\subsubsection{Base architecture}
\begin{code}
\captionof{listing}{Architecure of the Instruction Fetch}
\label{code:IF_code}   
\inputminted[fontsize=\footnotesize]{vhdl}{\srcfolder instr_fetch.vhd}
\end{code}
\newpage


\subsubsection{Testbench}
\begin{code}
\captionof{listing}{first Instruction Fetch testbench}
\label{code:IF_TB}
\inputminted[fontsize=\footnotesize]{vhdl}{\simfolder IF_testbench.vhd}
\end{code}
\newpage

\subsubsection{Instruction Decode [RV32I]}
\begin{code}
\captionof{listing}{Architecure of the Instruction Fetch}
\label{code:ID_code}  
\inputminted[fontsize=\footnotesize]{vhdl}{\srcfolder instr_decode.vhd}
\end{code}

\subsubsection{Decoder}
\begin{code}
\captionof{listing}{Decoder fully implementing RV32I instructions}
\label{code:ID_decoder}  
\inputminted[fontsize=\footnotesize]{vhdl}{\srcfolder decoder.vhd}
\end{code}

\subsubsection{Register File}
\begin{code}
\captionof{listing}{Simple 32 bits Register File}
\label{code:reg_file}  
\inputminted[fontsize=\footnotesize]{vhdl}{\srcfolder register_file.vhd}
\end{code}

\subsubsection{Testbench}
\begin{code}
\captionof{listing}{Instruction Decode Testbench}
\label{code:ID_TB}  
\inputminted[fontsize=\footnotesize]{vhdl}{\simfolder ID_testbench.vhd}
\end{code}
\newpage


\subsection{Instruction Execute [RV32I]}
\subsubsection{Architecture}
\begin{code}
\captionof{listing}{Architecture of Instruction Execute}
\label{code:IE_code}  
\inputminted[fontsize=\footnotesize]{vhdl}{\srcfolder instr_exec.vhd}
\end{code}

\subsubsection{ALU}
\begin{code}
\captionof{listing}{ALU}
\label{code:IE_ALU}  
\inputminted[fontsize=\footnotesize]{vhdl}{\srcfolder ALU.vhd}
\end{code}
\newpage


\subsubsection{Comparator}
\begin{code}
\captionof{listing}{Comparator}
\label{code:IE_comparator}  
\inputminted[fontsize=\footnotesize]{vhdl}{\srcfolder comparator.vhd}
\end{code}
\newpage


\subsubsection{Testbench}
\begin{code}
\captionof{listing}{Instruction Execute Testbench}
\label{code:IE_TB}  
\inputminted[fontsize=\footnotesize]{vhdl}{\simfolder IE_testbench.vhd}
\end{code}
\newpage


\subsection{Data Memory [RV32I]}
\begin{code}
\captionof{listing}{Data Memory stage}
\label{code:DM_code}  
\inputminted[fontsize=\footnotesize]{vhdl}{\srcfolder data_memory.vhd}
\end{code}


\subsection{WriteBack [RV32I]}
\begin{code}
\captionof{listing}{Write Back stage}
\label{code:WB_code}  
\inputminted[fontsize=\footnotesize]{vhdl}{\srcfolder write_back.vhd}
\end{code}

\subsection{Unpipelined Datapath Simulation}
\begin{code}
\captionof{listing}{Testbench with all components of the RV32I datapath}
\label{code:DPP_TB}  
\inputminted[fontsize=\footnotesize]{vhdl}{\simfolder WB_testbench.vhd}
\end{code}
\newpage

\section{Appendix C: Code segments of the finalized datapath}

\subsection{Final Instruction Fetch entity description}
\begin{code}
\captionof{listing}{Instruction Fetch implemented for a pipelined architecture}
\label{code:IF_PPLND}  
\inputminted[fontsize=\footnotesize]{vhdl}{\srcfolder instr_fetch_pipeline.vhd}
\end{code}

\subsection{Final Instruction Decode entity description}
\begin{code}
\captionof{listing}{Instruction Decode implemented for a pipelined architecture}
\label{code:ID_PPLND}  
\inputminted[fontsize=\footnotesize]{vhdl}{\srcfolder instr_decode_pipeline.vhd}
\end{code}

\subsection{Final Instruction Execute entity description}
\begin{code}
\captionof{listing}{Instruction Execute implemented for a pipelined architecture}
\label{code:IE_PPLND}  
\inputminted[fontsize=\footnotesize]{vhdl}{\srcfolder instr_exec_pipeline.vhd}
\end{code}

\subsection{Final Data Memory stage entity description}
\begin{code}
\captionof{listing}{Data Memory implemented for a pipelined architecture}
\label{code:DM_PPLND}  
\inputminted[fontsize=\footnotesize]{vhdl}{\srcfolder data_memory_pipeline.vhd}
\end{code}

\subsection{IF/ID register}
\begin{code}
\captionof{listing}{IF/ID register}
\label{code:IF_ID}  
\inputminted[fontsize=\footnotesize]{vhdl}{\srcfolder IF_ID.vhd}
\end{code}

\subsection{ID/IE register}
\begin{code}
\captionof{listing}{ID/IE register}
\label{code:ID_IE}  
\inputminted[fontsize=\footnotesize]{vhdl}{\srcfolder ID_IE.vhd}
\end{code}

\subsection{IE/DM register}
\begin{code}
\captionof{listing}{IE/DM register}
\label{code:IE_DM}  
\inputminted[fontsize=\footnotesize]{vhdl}{\srcfolder IE_DM.vhd}
\end{code}

\subsection{DM/WB register}
\begin{code}
\captionof{listing}{DM/WB register}
\label{code:DM_WB}  
\inputminted[fontsize=\footnotesize]{vhdl}{\srcfolder DM_WB.vhd}
\end{code}

\subsection{Pipelined architecture (Structural definition)}
\begin{code}
\captionof{listing}{Structural definition of a pipelined datapath}
\label{code:DTPTH_pipelined}  
\inputminted[fontsize=\footnotesize]{vhdl}{\srcfolder datapath.vhd}
\end{code}


\section{Appendix D: Testing Block for Nexys 4}

\subsection{Reading from register file and DM}
\begin{code}
\captionof{listing}{Architecture of a register and DM reader with a 7-segments display driver for Nexys 4 DDR}
\label{code:mem_read}  
\inputminted[fontsize=\footnotesize]{vhdl}{\srcfolder memory_reader.vhd}
\end{code}
